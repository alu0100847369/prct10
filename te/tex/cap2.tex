%%%%%%%%%%%%%%%%%%%%%%%%%%%%%%%%%%%%%%%%%%%%%%%%%%%%%%%%%%%%%%%%%%%%%%%%%%%%%%%
% Chapter 2: Fundamentos Te�ricos 
%%%%%%%%%%%%%%%%%%%%%%%%%%%%%%%%%%%%%%%%%%%%%%%%%%%%%%%%%%%%%%%%%%%%%%%%%%%%%%%

%++++++++++++++++++++++++++++++++++++++++++++++++++++++++++++++++++++++++++++++
\pi (pi) es la relaci�n entre la longitud de una circunferencia y su di�metro, en geometr�a euclidiana. Es un n�mero irracional y una de las constantes matem�ticas m�s importantes. Se emplea frecuentemente en matem�ticas, f�sica e ingiener�a. El valor num�rico de \pi, truncado a sus primeras cifras, es el siguiente:
{\pi}=3,14159265358979323846...
El valor de \pi se ha obtenido con diversas aproximaciones a lo largo de la historia, siendo una de las constantes matem�ticas que m�s aparece en las ecuaciones, junto con el n�mero e. Cabe destacar que el cociente entre la longitud de cualquier circunferencia y la de su di�metro no es constante en geometr�as no eucl�deas.

%++++++++++++++++++++++++++++++++++++++++++++++++++++++++++++++++++++++++++++++

\section{Primer apartado del segundo cap�tulo}
\label{2:sec:1}
  Primer p�rrafo de la primera secci�n.

\section{Segundo apartado del segundo cap�tulo}
\label{2:sec:2}
  Primer p�rrafo de la segunda secci�n.

